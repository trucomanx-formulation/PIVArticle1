The measurements of deformations in solid surfaces, mainly in structures used in the 
construction industry are of high complexity and demand expensive equipment. This topic is 
currently the object of research with the goal to improve the used techniques, to reduce the 
cost of testing, to promote fast and accurate measurement and to create new testing 
methodologies.%\textcolor{red}{Falta referencia}

The known methodologies may be divided into destructive and non-destructive approaches.
The first one deals with the tests that use the direct contact with the material under 
study, as it happens in the tests executed in Universal Testing Machines ($UTM$). 
Nondestructive testing techniques are those that use physical principles to infer the properties and 
behavior of materials \cite{Pereira2017}. 

The conventional analysis \cite{GOUVEA} 
presents drawbacks such as a high time consuming and number of samples required,
and consequently a high cost of operation. Thus, 
the increase of the use of non-destructive testing techniques is due to the reliability of obtained 
results, with quicker tests  that it does not harm  analyzed materials \cite{DEPAULA}.

Among the non-destructive techniques, optical techniques have become more popular 
in the field of displacement measurement  and deformation in solid bodies. The Particle 
Image Velocimetry ($PIV$) technique is an optical technique with great potential to measure 
displacements and trajectories of particles in a determined body or environment \cite{Pereira2017}.

\begin{sloppypar}
The $PIV$ technique was originally developed in the field of fluids and gases, with 
several applications in this field of study \cite{BANGALEE,XU}. 
Recently, some authors have studied the application of this methodology in solid materials to 
verify deformations and to find the mechanical properties \cite{BRAGAJUNIOR,MAGALHAES,PEREIRA,SOUZA}.
\end{sloppypar}

The main parameter for the execution of the $PIV$ technique is the type of pattern used 
and its distribution on the surface of the sample. The size of the square region of analysis,
 the distance that the algorithm should search a given region of 
analysis from a pre-established point, the search step and the degree of similarity 
between regions of analysis are also important parameters.

This work aims to evaluate the parameters related to the implementation of the $PIV$ 
technique such as the best type of pattern along with its distribution and randomness on the 
surface of the objects under study. In the same way, the parameters directly related to the $PIV$ 
algorithm are studied. These parameters 
are of fundamental importance to the measurement values discriminated by the $PIV$ technique 
to be as accurate as the values obtained by the conventional techniques of test and measurement.

The first section is the introduction and have the basic theory to understand the
$PIV$ analysis, additionally are defined some $PIV$  parameters that will be analyzed.
The second section shows the system setup to the analysis
of the effect of $PIV$ parameters in the $PIV$ procedure.
In the third section is showed the criteria for the selection the $PIV$
parameters. 
%%%%%%
The fourth section show the numerical results of tests presented 
in the second section.
%%%%%%
The fifth section  present the conclusion of article and some considerations.
The last section is a appendix and present the description of used algorithms.