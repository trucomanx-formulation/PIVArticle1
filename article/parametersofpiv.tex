\subsection{Parameters of PIV}
\label{sec:systemdesc}

\subsubsection{Analysis region}
An analysis region ($AR$), is a rectangular or square
image portion that is selected to analyze particle groups in the image.
Thus, if two regions are selected, they can be compared to each other using the correlation coefficient.


\subsubsection{Window size}
The window size ($WSIZE$) of a square analysis region
has reference to the length in pixels of the side of this region.
An increment in the $WSIZE$ value promotes an increase of the computational cost of 
method, but in counterpart, an increment in the quantity of the signal information
inside the analysis region, and consequently it decreases the probability of having
a false positive in the match of two analysis region in two images.
On the other hand, small $WSIZE$ values, avoid the loss of correlation 
between two matched analysis regions that undergo deformation.

\subsubsection{Threshold value}
The threshold value ($T$) used in the $PIV$ technique, indicates the minimum value
of $PCC$ that is used to detect a match between two analysis regions.
A high threshold value guarantees that the particle groups in matched 
analysis regions are effectively, the same; but it avoids, that two of the same 
analysis regions be matched when these suffer position deformations in its
particles groups.
Moreover, a low threshold value guarantees that the analysis regions 
can be matched after going through considerable  position deformations in 
its particles groups but widens the possibility of having false positives 
in the matched analysis regions.

\subsubsection{Step length}
The step length ($l_0$) is an integer value in pixels, which is used as minimum dislocation unit 
to choose the portion of the image where the analysis region will be selected from.
Thus, if two analysis regions are selected in two different images, if a match
is not declared between these images, one of these images is discarded and
a new analysis region is selected, distant a $l_0$ (horizontal or vertical) next from
discarded analysis region.


\subsubsection{Search length}
In the same context of the ``Step Length'', the search length ($L$) denotes the maximum
distance (horizontal and vertical) in pixels to select a new  analysis region, 
from the original position of the initial analysis region. Hence 
the distance displaced $nl_0, \forall n\in Z^+,$ to choose a new analysis region,
should be fulfill that $nl_0\leq L$.
