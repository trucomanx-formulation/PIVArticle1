The use of particle image velocimetry ($PIV$) technique for flow measurement
in liquid and gaseous materials and in solid deformations 
is an option over conventionally used techniques. This technique, for example, 
can be used in measurements and monitoring of structural parts under load 
with the advantage to be applied in field. However,
$PIV$ technique requires an adequate choice of all parameters to obtain accurate and reliable results. 
The objective of this study is 
the detailed analysis of window size, step size, search size, threshold and 
marker patterns, and their influence  on 
the final result. The study was carried out using test samples of Eucalyptus 
grandis wood subjected to static bending test in a universal testing machine. 
The application of the $PIV$ technique occurred simultaneously to the static 
bending tests. Dial indicators were placed in three regions of the test samples 
for comparison of the deformation values ​​obtained by the two techniques. The 
results showed that the adequate choice of the parameters of  $PIV$ technique 
contributed directly to the non-existence of false positives, increasing its 
precision and reliability.
%\textcolor{red}{It was concluded through this study that the $PIV$
%technique is a tool for measuring deformations with precision similar to 
%conventional test techniques, from the definition and choice of parameters 
%as established in the theoretical models}.