
Through the comparison of the traditional method 
to analyze load and break study of beams, using universal testing machine, 
to the technique using the particle image velocimetry method,
we observed similarity in the results when the adequate parameters criteria
to choose them were used.

The results shown in this paper demonstrated the importance of quality tests,
these are the displacement and rotation tests. Failure to observe these parameters
widens the possibility of false positives in the $PIV$ technique, and consequently
in the result of load and break study of beams.