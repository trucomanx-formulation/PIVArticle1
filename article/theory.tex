\subsection{Theory}
\subsubsection{Pearson Correlation Coefficient}
The Pearson Correlation Coefficient ($PCC$) \cite{Pearson} measures the correlation degree between two
signals $X$ and $Y$, with $N$ samples denoted by $X_i$ and $Y_i$ respectively.
\begin{equation}
 \rho(X,Y)=\frac{\sum\limits_{i=1}^{N} (X_i-\mu_{X})(Y_i-\mu_{Y})}{N~\sigma_{X}~\sigma_{Y}},
\end{equation} 
where $\mu_{X}$, $\sigma_{X}$, $\mu_{Y}$ and $\sigma_{Y}$ are the expected values and 
the standard deviation values of the signals $X$ and $Y$ respectively.

The range value of the coefficient is between $-1 \leq \rho(X,Y) \leq +1$, where a value close
to $+1$ indicates that there is a high correlation level between $X$ and $Y$, 
on the other hand values close to $0$ indicate that there isn't a correlation and 
finally values close to $-1$ indicate that there is a correlation but with opposite sense,
this means, for example, that when a variable grows the another decreases its value proportionally.

\subsubsection{Particle Image Velocimetry}
The Particle Image Velocimetry ($PIV$) is an optical technique designed 
to flow visualization.
This method needs, to be executed, that the flow in study contain particles, in other words, 
visible elements in the flow; thus
pictures of flow movement are taken and the position of particle groups are
identified between images, forming  a velocity flow map with magnitude and direction \cite{piv1}.

To recognize two identical particle groups between two consecutive images,
many methods may be used, the most widely used method in published material is the
$PCC$ method; for this purpose analysis regions are selected in each image
and if the $PCC$ value, calculated over these data, 
exceeds a threshold, the match is declared and the particle groups
are identified, else ways, another analysis region is selected in  one image and the match test
is performed again.

%In this study line also can be found Particle Tracking Velocimetry (PTV)

